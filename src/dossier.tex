\documentclass[a4paper,12pt, draft]{report} 

\usepackage[francais]{babel}
\usepackage[utf8]{inputenc}
\usepackage[T1]{fontenc}
\author{Bérenger Ossete Gombe}
\title{Dossier PPP}

\begin{document}
\maketitle

\newpage
\tableofcontents{}
\newpage

\part{Introduction}
\chapter{Remerciements}
\chapter{Guide de lecture}

\part{Bilan personnel}

\chapter{Compétences et savoirs} 

Dans ce chapitre nous allons faire une synthèse de mes compétences.
Celles-ci seront traitées selon leurs natures.
Étant issue de savoir différents, j'ai décidé de les traiter en deux parties.


\section{Introspection}
Dans cette partie nous allons nous intéresser aux compétences liées au \textbf{savoir-être} et à la \textbf{personnalité}. Celle-ci est consituée de traits que l'on peut considérer comme des qualités ou bien comme des défauts. Il est alors difficile de rester objectif tant cela est une affaire de \textbf{point de vue}. Je me vois d'une certaine manière, les autres me voient différement, et je me vois à travers le regard des autres encore différement. Cette divergence des points de vues rend mon travail d'analyse tout à fait \textbf{subjectif}. Je traiterai ici uniquement ma vision que je m'efforcerai de pondéré en m'aidant d'un exercice d'autoportrait\footnote{réference nécessaire} effectué dans le cadre du CMI\footnote{Cursus Master en Ingénierie}.

\subsection{Comportement}
\begin{figure}[h]
  \begin{tabular}{|l|l|l|}
    \hline
  \item \texttt{l} \textsc{Trait} & \textsc{Qualité} & \textsc{Défaut}\\
    \hline
    \hline
  \item Curieux & Passionné & Trop agité, éparpillé \\
    \hline
    
  \end{tabular}\\
\caption{une liste de}
\end{figure}

\begin{description}
\item [Curieux]car j'adore apprendre et découvir de nouvelles choses.
\item [Partageur]car j'aime recevoir autant qu'offrir et je crois en la coopération.
\item [Volontaire]car j'aime relever des challenges, aller plus loin et appronfondir.
\end{description}

\begin{description}
\item [Timide]J'ai du mal à faire face à certaines situations.
\item [Perfectionniste]Je m'acharne souvent par refus de l'abandon et de l'échec.
\item [Rêveur]J'ai tendance à suivre mon instinct et mes envies.
\end{description}

\subsection{Motivations}
\subsection{Valeurs}

\section{Compétences métiers} 
\subsection{Ma formation}
\subsection{Mon expérience}

\chapter{Synthèse}
\section{Mes objectifs}
\section{Mes priorités}

\part{Projet Professionnel}

\chapter{Construction du projet} 
\section{Missions et tâches}
\section{L'environnement}

\chapter{Vers un métier}
\section{Le domaine d'activité}
\section{La fonction}

\chapter{Rencontre avec des professionnels}
\section{Le monde du libre}
\section{Le monde de l'enseignement et de la recherche}

\chapter{Confrontation au marché}
\section{Les besoins}
\section{Les profils recherchés}
\section{Mes atouts}

\part{Stratégies d'action}

\chapter{Stratégie d'acquisition de compétences}
\section{La formation}
\section{Les expériences}
\subsection{Du choix des stages}

\chapter{Stratégie d'adaptation au marché}
\section{Secteurs}
\section{Entreprises}

\chapter{Stratégie de communication}
\section{Les réseaux professionnels}
\section{Recherche de stages et d'emplois}
\subsection{Lettre de motivation}
\subsection{CV}
\subsection{Entretiens}

\part{Conclusion}
\chapter{Apport du Cursus Master en Ingénerie}


\end{document} 
